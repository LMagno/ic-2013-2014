%%%%%%%%%%%%%%%%%%%%%%%%%%%%%%
%  This Beamer template was created by Cameron Bracken.
%  Anyone can freely use or modify it for any purpose
%  without attribution.
%
%  Last Modified: January 9, 2009
%

\documentclass[xcolor=x11names,compress]{beamer}

% General document %%%%%%%%%%%%%%%%%
\usepackage{graphicx}
\usepackage{tikz}
\usepackage[brazilian]{babel}
\usepackage[utf8]{inputenc}
\usepackage[T1]{fontenc}
\usepackage{mathtools}
\usepackage{amsfonts}
\usepackage{graphicx}
\usepackage{booktabs}
\usepackage{url}
\usetikzlibrary{decorations.fractals}
%%%%%%%%%%%%%%%%%%%%%%%%%%%

% Beamer Layout %%%%%%%%%%%%%%%%%
\useoutertheme[subsection=false,shadow]{miniframes}
\useinnertheme{default}
\usefonttheme[mathserif]{serif}
\usepackage{palatino}

\setbeamerfont{title like}{shape=\scshape}
\setbeamerfont{frametitle}{shape=\scshape}

\definecolor{azulzinho}{RGB}{248,249,254}
\setbeamercolor*{background canvas}{bg=azulzinho}
\setbeamercolor*{lower separation line head}{bg=DeepSkyBlue4}
\setbeamercolor*{normal text}{fg=black,bg=white}
\setbeamercolor*{alerted text}{fg=red}
\setbeamercolor*{example text}{fg=black}
\setbeamercolor*{structure}{fg=black}

\setbeamercolor*{palette tertiary}{fg=black,bg=black!10}
\setbeamercolor*{palette quaternary}{fg=black,bg=black!10}

\renewcommand{\(}{\begin{columns}}
\renewcommand{\)}{\end{columns}}
\newcommand{\<}[1]{\begin{column}{#1}}
\renewcommand{\>}{\end{column}}
%%%%%%%%%%%%%%%%%%%%%%%%%



\title{Uma Ferramenta de Software para a Predição de Desempenho de Workflows Científicos}
\author{Lucas Magno\inst{1} \\ Kelly Rosa Braghetto\inst{2}}
\institute{\inst{1} Instituto de Física \\ \inst{2} Instituto de Matemática e Estatística \\[0.2cm] Universidade de São Paulo}
\date{SIICUSP 2014}

\begin{document}


%%%%%%%%%%%%%%%%%%%%%%%%%%%
%%%%%%%%%%%%%%%%%%%%%%%%%%%
\begin{frame}
    \titlepage
\end{frame}

\section{\scshape Motivação}

    \subsection{Introdução}
        \begin{frame}{Introdução}
            \begin{itemize}
                \item<1-> Workflows científicos
                \item<2-> Custo de execução
                \item<3-> Previsão de desempenho
                \item<4-> Modelagem analítica
                    \begin{itemize}
                        \item<4-> Redes de Petri
                        \item<4-> Álgebras de processos
                    \end{itemize}
            \end{itemize}
        \end{frame}

    \subsection{Dificuldades}
        \begin{frame}{Dificuldades}
            \begin{itemize}[<+->]
                \item Linguagens e modelos estocásticos
                \item Programas de simulação e análise numérica
                \item Diversas áreas da ciência
            \end{itemize}
        \end{frame}

    \subsection{Objetivos}
        \begin{frame}{Objetivos}
            \begin{itemize}[<+->]
                \item Programa em \emph{Python}
                    \begin{itemize}
                        \item Descrição textual simples e intuitiva do workflow
                        \item Geração do modelo analítico
                        \item Extração dos índices de desempenho
                    \end{itemize}
            \end{itemize}
        \end{frame}

\section{\scshape Métodos}

    \subsection{Descrição do Workflow}
        \begin{frame}{Descrição do Workflow}

        \end{frame}

    \subsection{Leitura do Workflow de Entrada}
        \begin{frame}{Leitura do Workflow de Entrada}

        \end{frame}

    \subsection{Estrutura de Dados baseada em Grafo}
        \begin{frame}{Estrutura de Dados baseada em Grafo}

        \end{frame}

    \subsection{Visualização do Workflow}
        \begin{frame}{Visualização do Workflow}

        \end{frame}

    \subsection{Modelagem Analítica}
        \begin{frame}{Modelagem Analítica}

        \end{frame}

    \subsection{Extração dos Índices de Desempenho}
        \begin{frame}{Extração dos Índices de Desempenho}

        \end{frame}

\section{\scshape Resultados}

    \subsection{Descrição do Workflow}
        \begin{frame}{Descrição do Workflow}

        \end{frame}

    \subsection{Visualização do Workflow}
        \begin{frame}{Visualização do Workflow}

        \end{frame}

    \subsection{Modelagem Analítica}
        \begin{frame}{Modelagem Analítica}

        \end{frame}

    \subsection{Extração dos Índices de Desempenho}
        \begin{frame}{Extração dos Índices de Desempenho}

        \end{frame}

\end{document}
