%%%%%%%%%%%%%%%%%%%%%%%%%%%%%%
%  This Beamer template was created by Cameron Bracken.
%  Anyone can freely use or modify it for any purpose
%  without attribution.
%
%  Last Modified: January 9, 2009
%

\documentclass[xcolor=x11names,compress]{beamer}

% General document %%%%%%%%%%%%%%%%%
\usepackage{graphicx}
\usepackage{tikz}
\usepackage[brazilian]{babel}
\usepackage[utf8]{inputenc}
\usepackage[T1]{fontenc}
\usepackage{mathtools}
\usepackage{amsfonts}
\usepackage{graphicx}
\usepackage{booktabs}
\usepackage{url}
\usepackage{listings}
\usepackage{verbatim}
\usetikzlibrary{decorations.fractals}
%%%%%%%%%%%%%%%%%%%%%%%%%%%

% Beamer Layout %%%%%%%%%%%%%%%%%
\useoutertheme[subsection=false,shadow]{miniframes}
\useinnertheme{default}
\usefonttheme[mathserif]{serif}
\usepackage{palatino}

\setbeamerfont{title like}{shape=\scshape}
\setbeamerfont{frametitle}{shape=\scshape}

\definecolor{azulzinho}{RGB}{248,249,254}
\setbeamercolor{footlinecolor}{bg=azulzinho}
\setbeamercolor*{background canvas}{bg=azulzinho}
\setbeamercolor*{lower separation line head}{bg=DeepSkyBlue4}
\setbeamercolor*{normal text}{fg=black,bg=white}
\setbeamercolor*{alerted text}{fg=red}
\setbeamercolor*{example text}{fg=black}
\setbeamercolor*{structure}{fg=black}

\setbeamercolor*{palette tertiary}{fg=black,bg=black!10}
\setbeamercolor*{palette quaternary}{fg=black,bg=black!10}

\renewcommand{\(}{\begin{columns}}
\renewcommand{\)}{\end{columns}}
\newcommand{\<}[1]{\begin{column}{#1}}
\renewcommand{\>}{\end{column}}

\beamertemplatenavigationsymbolsempty
\setbeamertemplate{footline}{%
    \leavevmode%
    \hbox{%
        \begin{beamercolorbox}[wd=\paperwidth, ht=2.25ex, dp=1ex, right]{footlinecolor}%
            \usebeamerfont{title in head/foot} SIICUSP 2014 – 22º Simpósio Internacional de Iniciação Científica e Tecnológica da USP%
            \hspace{0.3cm}
        \end{beamercolorbox}%
    }%
    \vskip0pt%
}

\lstset{numbers=left,
        numberstyle=\tiny,
        showstringspaces=false,
}
%%%%%%%%%%%%%%%%%%%%%%%%%



\title{Uma Ferramenta de Software para a Predição de Desempenho de Workflows Científicos}
\author{Lucas Magno\inst{1} \\ Kelly Rosa Braghetto\inst{2}}
\institute{\inst{1} Instituto de Física \\ \inst{2} Instituto de Matemática e Estatística \\[0.2cm] Universidade de São Paulo}
\date{PIBIC/CNPq}

\begin{document}


%%%%%%%%%%%%%%%%%%%%%%%%%%%
%%%%%%%%%%%%%%%%%%%%%%%%%%%
\begin{frame}
    \titlepage
\end{frame}

\section{\scshape Motivação}

    \subsection{Introdução}
        \begin{frame}{Introdução}
            \begin{itemize}
                \item<1-> Workflows científicos
                \item<2-> Custo de execução
                \item<3-> Previsão de desempenho
                \item<4-> Modelagem analítica
                    \begin{itemize}
                        \item<4-> Redes de Petri
                        \item<4-> Álgebras de processos
                    \end{itemize}
            \end{itemize}
        \end{frame}

    \subsection{Dificuldades}
        \begin{frame}{Dificuldades}
            \begin{itemize}[<+->]
                \item Linguagens e modelos estocásticos
                \item Programas de simulação e análise numérica
                \item Diversas áreas da ciência
            \end{itemize}
        \end{frame}

    \subsection{Objetivos}
        \begin{frame}{Objetivos}
            \begin{itemize}[<+->]
                \item Ferramenta de software
                    \begin{itemize}
                        \item Descrição simples do workflow
                        \item Geração do modelo analítico
                        \item Extração dos índices de desempenho
                    \end{itemize}
            \end{itemize}
        \end{frame}

\section{\scshape Métodos e Resultados}

    \subsection{O Programa}
        \begin{frame}{O Programa}
            \begin{itemize}
                \item<1-> \texttt{wkf2pepa}
                \item<2-> \emph{Python}
                    \begin{itemize}
                        \item<3-> Alto nível
                        \item<3-> Bibliotecas
                    \end{itemize}
            \end{itemize}
        \end{frame}

    \subsection{Descrição do Workflow}
        \begin{frame}{Descrição do Workflow}
            \begin{itemize}[<+->]
                \item Linguagem textual simples e intuitiva
                \item Baseada na linguagem \emph{DOT}
                \item Grafos direcionados
            \end{itemize}
        \end{frame}

    %\subsection{Descrição do Workflow, exemplo}
\begin{frame}[fragile]{Descrição do Workflow}
    \framesubtitle{Exemplo}
    \begin{lstlisting}[basicstyle=\tiny]
digraph {
    a          -> b;
    b          -> and1;
    and1 [and] -> e, xor1;
    xor1 [xor] -> [0.15] c, [0.85] d;
    e    [0.5] -> and2;
    c          -> xor2;
    d          -> xor2;
    xor2       -> and2;
    and2       -> f;
}
    \end{lstlisting}
\end{frame}

    \subsection{Leitura do Workflow de Entrada}
        \begin{frame}{Leitura do Workflow de Entrada}
            \begin{itemize}[<+->]
                \item Analisadores léxico (\emph{lexer}) e sintático (\emph{parser})
                \item PLY - \emph{Python Lex-Yacc}
            \end{itemize}
        \end{frame}

    \subsection{Estrutura de Dados na Memória}
        \begin{frame}{Estrutura de Dados na Memória}
            \begin{itemize}
                \item<1-> Grafo
                    \begin{itemize}
                        \item<1-> Generalidade
                        \item<1-> Independência de linguagem
                    \end{itemize}
                \item<2-> Classes
                    \begin{itemize}
                        \item<2-> Flexibilidade
                        \item<2-> Clareza
                        \item<3-> \emph{Node, Edge, Workflow}
                    \end{itemize}
            \end{itemize}
        \end{frame}

    \subsection{Visualização do Workflow}
        \begin{frame}{Visualização do Workflow}
            \begin{itemize}[<+->]
                \item Verificação da estrutura em memória
                \item Linguagem \emph{DOT}
                \item \emph{PDF}
            \end{itemize}
        \end{frame}

\begin{frame}[fragile]{Visualização do Workflow}
\framesubtitle{Exemplo}
    \begin{columns}[c]
        \begin{column}[l]{5cm}
        \begin{lstlisting}[basicstyle=\tiny]
digraph workflow1 {
    a    [shape=box, label=a];
    xor1 [shape=diamond,label=xor1,color=green];
    c    [shape=box,label=c];
    b    [shape=box,label=b];
    e    [shape=box,label=e];
    d    [shape=box,label=d];
    f    [shape=box,label=f];
    xor2 [shape=diamond,label=xor2,color=green];
    and1 [shape=diamond,label=and1,color=blue];
    and2 [shape=diamond,label=and2,color=blue];
    and1 -> e;
    and2 -> f;
    c    -> xor2;
    e    -> and2;
    xor2 -> and2;
    xor1 -> d [label="0.85"];
    and1 -> xor1;
    b    -> and1;
    a    -> b;
    xor1 -> c [label="0.15"];
    d    -> xor2;
}
        \end{lstlisting}
        \end{column}

        \begin{column}{5cm}
            \hfill
            \includegraphics[width=2.9cm]{workflow1.pdf}
        \end{column}
    \end{columns}
\end{frame}

    \subsection{Modelagem Analítica}
        \begin{frame}{Modelagem Analítica}
            \begin{itemize}
                \item<1-> Álgebra de processos estocástica
                    \begin{itemize}
                        \item<2-> Composicionalidade
                        \item<2-> Formalismo
                        \item<2-> Abstração
                    \end{itemize}
                \item<3-> \emph{PEPA - Performance Evaluation Process Algebra}
                    \begin{itemize}
                        \item<4-> Bem desenvolvida
                        \item<4-> Ferramentas de apoio
                        \item<4-> Cadeia de Markov
                    \end{itemize}
            \end{itemize}
        \end{frame}

\begin{frame}[fragile]{Modelagem Analítica}
    \framesubtitle{Exemplo}
    \begin{lstlisting}[basicstyle=\tiny]
r_a = 1.0; r_b = 1.0; r_e = 0.5;
r_c = 1.0; r_d = 1.0; r_f = 1.0;

r_AND = 100.0; r_XOR = 100.0; r_OR  = 100.0;

prob_xor1_c = 0.15;
prob_xor1_d = 0.85;

r_xor1_c = prob_xor1_c * r_XOR;
r_xor1_d = prob_xor1_d * r_XOR;

P = (a, r_a) . (b, r_b) . (and1, r_AND) . (and2, r_AND) . (f, r_f) . P;

P_and1_e = (and1, r_AND) . (e, r_e) . (and2, r_AND) . P_and1_e;
P_and1_xor1 = (and1, r_AND) . P_xor1;
P_xor1_c = (c, r_c) . P_xor2;
P_xor1_d = (d, r_d) . P_xor2;
P_xor1 = (xor1, r_xor1_c) . P_xor1_c + (xor1, r_xor1_d) . P_xor1_d;
P_xor2 = (xor2, r_XOR) . (and2, r_AND) . P_and1_xor1;

P <and1, and2> (P_and1_e <and1, and2> P_and1_xor1)
    \end{lstlisting}
\end{frame}

    \subsection{Extração dos Índices de Desempenho}
        \begin{frame}{Extração dos Índices de Desempenho}
            \begin{itemize}[<+->]
                \item \emph{pyPEPA}
                \item Probabilidades dos estados no regime estacionário
                \item Rendimentos (\emph{throughput})
                \item Taxas de utilização
            \end{itemize}
        \end{frame}

\begin{frame}[fragile]{Extração dos Índices de Desempenho}
    \framesubtitle{Exemplo}
    \begin{lstlisting}[basicstyle=\tiny]
Throughput:
xor1   -2.62364490393e-16
c        9.38551843388e-18
xor2     7.5563162709e-20
e        0.49504950495
and1     0.49504950495
and2     0.49504950495

Utilization:
P                  1.0

P_and1_e           0.0049504950495
(and2              0.0049504950495
(e                 0.990099009901

P_xor1_d           0.0049504950495
P_xor1_c           3.43509599203e-20
P_xor1           -6.60131287319e-19
r_AND).P_and1_e    0.0049504950495
P_xor2             3.49163606874e-22
(and2              4.07395329479e-22
r_e).((and2        0.990099009901
P_and1_xor1      -1.31596606031e-18
    \end{lstlisting}
\end{frame}

\section{\scshape Conclusão}

    \subsection{Conclusão}
        \begin{frame}{Conclusão}
            \begin{itemize}[<+->]
                \item Ferramenta de software \texttt{wkf2pepa}
                \item Descrição textual simples de workflows comuns
                \item Extração de índices de desempenho de forma automática
                \item Não requer conhecimento de detalhes sobre modelagem estocástica e análise numérica
            \end{itemize}
        \end{frame}

    \subsection{Trabalhos Futuros}
        \begin{frame}{Trabalhos Futuros}
            \begin{itemize}[<+->]
                \item Operador \emph{OR}
                \item Workflows mais complexos
                \item Descrição dos recursos disponíveis
                \item Comparação com índices de desempenho obtidos por outras ferramentas
            \end{itemize}
        \end{frame}

    \subsection{Agradecimentos}
        \begin{frame}{Obrigado!}

        \end{frame}

\end{document}
