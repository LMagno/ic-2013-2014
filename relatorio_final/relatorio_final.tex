\documentclass[a4paper,11pt]{article}
\usepackage{listingsutf8}
\usepackage{color}
\usepackage{scalefnt}
\usepackage{cancel}
\usepackage{picinpar}
\usepackage{subfig}
\usepackage{graphicx}
\usepackage{amssymb}
\usepackage{amsmath}
\usepackage[brazilian]{babel}
\usepackage[utf8]{inputenc}
\usepackage[T1]{fontenc}
\usepackage{setspace}
\usepackage{caption}
\usepackage[left=2.5cm,right=2.5cm,top=2.0cm,bottom=1.5cm]{geometry}
\usepackage{siunitx}
\usepackage[colorlinks=true,linkcolor=black,linktoc=all]{hyperref}
\usepackage{indentfirst}
\usepackage[nottoc]{tocbibind}
\usepackage[fixlanguage]{babelbib}
\selectbiblanguage{brazilian}
\renewcommand{\lstlistingname}{Código}
\lstset{
	inputencoding=utf8,
	extendedchars=true,
	numbers=left,
	numberstyle=\tiny,
	frame=lines,
	captionpos=b,
	literate=
		{á}{{\'a}}1 {é}{{\'e}}1 {í}{{\'i}}1 {ó}{{\'o}}1 {ú}{{\'u}}1 {ù}{{\`u}}1%
		{Á}{{\'A}}1 {É}{{\'E}}1 {Í}{{\'I}}1 {Ó}{{\'O}}1 {Ú}{{\'U}}1%
		{à}{{\`a}}1 {è}{{\'e}}1 {ì}{{\`i}}1 {ò}{{\`o}}1 {ò}{{\`o}}1%
		{À}{{\`A}}1 {È}{{\'E}}1 {Ì}{{\`I}}1 {Ò}{{\`O}}1 {Ò}{{\`O}}1%
		{ä}{{\"a}}1 {ë}{{\"e}}1 {ï}{{\"i}}1 {ö}{{\"o}}1 {ü}{{\"u}}1%
		{Ä}{{\"A}}1 {Ë}{{\"E}}1 {Ï}{{\"I}}1 {Ö}{{\"O}}1 {Ü}{{\"U}}1%
		{â}{{\^a}}1 {ê}{{\^e}}1 {î}{{\^i}}1 {ô}{{\^o}}1 {û}{{\^u}}1%
		{Â}{{\^A}}1 {Ê}{{\^E}}1 {Î}{{\^I}}1 {Ô}{{\^O}}1 {Û}{{\^U}}1%
		{ã}{{\~a}}1 {ẽ}{{\~e}}1 {ĩ}{{\~i}}1 {õ}{{\~o}}1 {ũ}{{\~u}}1%
		{Ã}{{\~A}}1 {Ẽ}{{\~E}}1 {Ĩ}{{\~I}}1 {Õ}{{\~O}}1 {Ũ}{{\~U}}1%
		{œ}{{\oe}}1 {Œ}{{\OE}}1 {æ}{{\ae}}1 {Æ}{{\AE}}1 {ß}{{\ss}}1%
		{ç}{{\c c}}1 {Ç}{{\c C}}1 {ø}{{\o}}1 {å}{{\r a}}1 {Å}{{\r A}}1%
		{€}{{\EUR}}1 {£}{{\pounds}}1,
}

\date{\vfill Junho de 2014}
\title{
	Relatório Final de Iniciação Científica \vspace{40pt}\\ 
	Uma Ferramenta de Software para a Predição de Desempenho de Workflows Científicos
}

\author{
	\vspace{100pt}\\
	Aluno: Lucas Magno \\ 
	Bolsista PIBIC do CNPq \\
	Instituto de Física (IF) \\
	\\ \\
	Orientadora: Kelly Rosa Braghetto\\ 
	Departamento de Ciência da Computação (DCC) \\ 
	Instituto de Matemática e Estatística (IME) \\
	\\ \\ \\
	Universidade de São Paulo
} 

\begin{document}
  \maketitle
  \thispagestyle{empty}
  \newpage
  \clearpage
  \setcounter{page}{1}
  \section*{Resumo}

    Este documento descreve as atividades realizadas durante o per\'iodo de julho de 2013 a junho de 2014 no \^ambito do projeto de iniciação científica do aluno Lucas Magno, n\'umero USP 7994983, orientado pela Profa. Dra. Kelly Rosa Braghetto e financiado por uma bolsa PIBIC/CNPq.

    O objetivo principal do projeto foi desenvolver uma ferramenta de software para a conversão autom\'atica de modelos de \emph{workflows} em modelos estocásticos na \'algebra de processos \emph{PEPA - Performance Evaluation Process Algebra}. A partir desses modelos estocásticos, \'e possível extrair prediç\~oes de desempenho de \emph{workflows}.
  \newpage
  \tableofcontents
  \newpage

  \section{Introdução}
  	Inicialmente desenvolvidos para automatizar processos industriais e empresariais, os \emph{workflows} se popularizaram e passaram a ser usados na modelagem e automatização de experimentos científicos em diversas áreas da ciência. Um \emph{workflow científico} é a descrição completa ou parcial de um experimento científico em termo de suas atividades, controles de fluxo e dependência de dados \cite{phd:gadelha12}. 

  	\subsection{Representação de \emph{Workflows} Científicos}
  	Há várias maneiras de se representar um \emph{workflow científico}, mas entre elas se destacam \cite{phd:oga11}:

  	\begin{itemize}
  		\item \emph{Grafos direcionados}: uma das formas mais comuns e simples de representação de \emph{workflows}, permitem sua visualização gráfica e facilitam sua descrição através de modelos gráficos. Num grafo, os nós representam as atividades de um experimento científico e, as arestas, as dependências entre essas.

  		\item \emph{\emph{Unified Modeling Language} (UML)}: linguagem padrão para modelagem de \emph{software} orientado a objetos, tendo como um de seus recursos o diagrama de atividades, que pode ser utilizado para descrever as dependências entre atividades e, portanto, \emph{workflows}. 

        \item \emph{Redes de Petri}: muito utilizadas para modelar comportamento concorrente em sistemas distribuídos discretos, podem ser interpretadas como um caso particular de grafos direcionados, diferindo destes por possuírem dois tipos de nós: lugares e transições. As arestas sempre conectam dois tipos diferentes de nós, tornando o grafo bipartido.

  		\item \emph{Álgebras de Processos}: podem ser entendidas como um estudo do comportamento de sistemas paralelos ou distribuídos por meio de uma abordagem algébrica, que permite verificações, análises algébricas e aperfeiçoamento de processos por meio de transformações. Sendo assim, diferentemente dos exemplos anteriores, não possuem representação gráfica, somente textual.
 		
  	\end{itemize}

    Os mecanismos de representação listados acima são usados para criar modelos que especificam a ordem de execução das atividades dos \emph{workflows}. Mas as redes de Petri e as \'algebras de processos são linguagens formais (a sem\^antica de seus construtores \'e definida de forma não amb\'igua). Por isso, elas permitem tamb\'em que se verifique propriedades qualitativas ou quantitativas dos modelos de \emph{workflow}, como, por exemplo, a exist\^encia de \emph{deadlocks} (pontos de impasse). 

  	No entanto, somente grafos direcionados e álgebras de processo serão utilizados neste trabalho.

    \newpage
  	\subsection{Composição de \emph{Workflows} Científicos}
        Um \emph{workflow} pode ser composto de diversos elementos, mas os relevantes neste projeto são atividades, que representam atividades reais de um experimento, e estruturas para descrever o fluxo de controle, que definem a ordem de execução das atividades dentro do \emph{workflow}. Temos como exemplos de estruturas \emph{sequência}, \emph{paralelismo}, \emph{escolha} e \emph{sincronização}.

        As estruturas de controle de fluxo são definidas nos modelos de workflows por meio de operadores. Os modelos de workflows considerados neste trabalho utilizam os seguintes operadores:
                
        \begin{itemize}
            \item \emph{AND} (paralelismo/sincronização): todos os ramos s\~ao executados
            \item \emph{XOR} (escolha exclusiva/junção): apenas um ramo \'e executado
            \item \emph{OR} (escolha múltipla/junção): um ou mais ramos s\~ao executados
        \end{itemize}

        Os ramos abertos por um operador (chamado operador de divis\~ao) terminam em um \'unico operador do mesmo tipo (chamado operador de junç\~ao), cuja \'unica funç\~ao \'e sincronizar o final dos ramos executados.

  	\subsection{Análise de Desempenho}
        É comum em experimentos científicos a manipulação de enormes quantidades de dados e processos muito demorados, o que estimula o cientista a aperfeiçoar o experimento antes de sua execução, pois esta pode demandar muitos recursos e tempo. Daí a necessidade da análise do desempenho de um \emph{workflow}, que pode ser feita através de três métodos \cite{phd:kelly11}:

  	\begin{itemize}
  		\item \emph{Medição:} consiste na execução do \emph{workflow} uma quantidade estatisticamente relevante de vezes e então no cálculo dos tempos médios de interesse. Logo, só pode ser aplicada a sistemas já implementados, não tendo caráter preditivo;

  		\item \emph{Simulação:} baseada em modelos matemáticos cuja solução é dada por um programa que simula o comportamento modelado;

  		\item \emph{Modelagem analítica:} também baseada em modelos, mas analisa numericamente determinados aspectos de interesse em um sistema.
  	\end{itemize}

  	Neste projeto, foi usada a modelagem analítica, por ser preditiva, rápida e não muito difícil de se implementar, embora menos precisa que a medição. Tanto as redes de Petri quanto as álgebras de processos são formalismos que possuem extensões estocásticas e que, portanto, podem ser usados no método modelagem analítica. 

  	Como álgebra de processos estocástica foi escolhida a PEPA, \emph{Performance Evaluation Process Algebra} \cite{web:pepa}, porque o uso desse formalismo ainda não foi profundamente explorado para a análise de desempenho preditiva de \emph{workflows} científicos.
  	
  \section{Objetivos e Contribuiç\~oes}

  	Uma desvantagem da modelagem analítica usando PEPA é a necessidade da descrição do \emph{workflow} em uma linguagem de modelagem estocástica e utilização de programas específicos para a análise, exigindo do usuário um certo nível de conhecimento sobre álgebras de processo. No entanto, \emph{workflows} científicos são utilizados em diversas áreas da ciência que não necessitam de um grande aprofundamento em computação, o que pode inviabilizar a aplicação deste método. 

    O objetivo do trabalho foi facilitar a extração de prediç\~oes de desempenho para \emph{workflows} cient\'ificos. Para isso, foi desenvolvida uma ferramenta de software que gera modelos estoc\'asticos em \emph{PEPA} e sua soluç\~ao num\'erica a partir de modelos de \emph{workflows}. Os modelos de \emph{workflows} usados como entrada para a ferramenta s\~ao descritos textualmente na forma de um grafo dirigido - uma representação simples e que pode ser usada com facilidade por usu\'arios n\~ao especialistas. Al\'em do modelo em \emph{PEPA} e sua soluç\~ao, a ferramenta tamb\'em gera uma representaç\~ao gr\'afica do modelo de \emph{workflow}, que permite que o usu\'ario possa verific\'a-lo mais facilmente.

  \newpage
  \section{Metodologia}

  	Para que possua um modelo correspondente em \emph{PEPA}, um modelo de \emph{workflow} precisa ser bem estruturado e n\~ao possuir ambiguidades sem\^anticas. Por essa raz\~ao, neste trabalho consideramos apenas modelos de \emph{workflow} que apresentam somente um ponto de entrada e um ponto de saída, têm sua estrutura em forma de ``blocos''  e não apresentam ciclos, ou laços, o que permite uma implementação mais simples.


  	\subsection{O Programa}
  		Para automatizar o processo de predição de desempenho, foi implementado um programa que realiza as seguintes etapas:

  		\begin{enumerate}
  			\item Lê como entrada uma descrição textual de um \emph{workflow}
  			\item Gera uma estrutura de dados baseada em grafo na memória representando o \emph{workflow}
  			\item Gera uma visualização do \emph{workflow} de entrada
  			\item Gera um modelo analítico (estocástico) do \emph{workflow}
            \item Soluç\~ao num\'erica do modelo anal\'itico e extraç\~ao dos \'indices de desempenho
  		\end{enumerate}

  		Para tanto, foi escolhida a linguagem \emph{Python}, por flexibilidade, facilidade de aprendizado e grande número de bibliotecas auxiliares.

  		Na etapa 1, foi definida uma gramática simples baseada na linguagem DOT \cite{web:dot} e se utilizou os analisadores léxico e sintático disponíveis na biblioteca PLY, \emph{Python Lex-Yacc} \cite{web:ply} (com dependência na biblioteca \emph{pyParsing} \cite{web:pyparsing}), para efetuar sua leitura.

        A implementaç\~ao da etapa 2 foi feita inicialmente com o aux\'ilio da biblioteca \emph{python-graph} \cite{web:python-graph}, que permite a criação e manipulação de diversos tipos de grafos por meio de classes. Entretanto, mais recentemente, o programa foi reescrito para n\~ao depender dessa biblioteca, definindo explicitamente, portanto, as classes de manipulaç\~ao de n\'os, arestas e \emph{workflows}. A implementação customizada permitiu maior flexibilidade e clareza de código, além de melhorar a performance e diminuir as dependências. No entanto, a etapa 3, que era realizada pela \emph{python-graph}, teve que ser manualmente implementada, criando a descrição do grafo em linguagem DOT e sua visualização gráfica a partir da biblioteca \emph{Graphviz} \cite{web:graphviz}.

        Na etapa 4, a partir da estrutura de dados definida na etapa 2, \'e criado o modelo analítico do grafo, na linguagem PEPA. Sua implementação não exigiu o uso de nenhuma biblioteca específica e funcionou como esperado nos casos testados, mas, devido a sua complexidade, mais testes serão necessários para confirmar sua eficiência. 

        A solução numérica do modelo analítico e a extraç\~ao de \'indices de desempenho foram realizadas atrav\'es da biblioteca \emph{pyPEPA} \cite{web:pypepa}, uma implementação recente da \emph{PEPA} em \emph{Python}.

    \newpage
	\subsection{Nota Técnica}

		O projeto foi desenvolvido e testado sob sistema operacional linux e \emph{Python} 2.7.
        Seu c\'odigo fonte, bem como exemplos de \emph{workflow} de entrada e suas representaç\~oes em \emph{PEPA} geradas podem ser conferidos na p\'agina do projeto \cite{web:script}.

        \subsubsection{Depend\^encias}
            Em um sistema linux baseado em \emph{Debian}, os seguintes pacotes devem ser instalados al\'em da instalaç\~ao padrão do \emph{python} 2.7:

            \begin{itemize}
                \item python-ply (Biblioteca Python Lex-Yacc)
                \item libgv-python (Biblioteca Graphviz)
                \item python-pyparsing (Biblioteca pyParsing)
                \item python-numpy (Biblioteca NumPy)
                \item python-scipy (Biblioteca SciPy)

            \end{itemize}

        \subsubsection{Execução}
            O programa deve ser executado a partir de um terminal utilizando os seguintes comandos:

            \lstinputlisting{comandos}

            Onde \$ representa que se est\'a num terminal, \emph{script.py} \'e o arquivo que cont\'em o programa, e \emph{input1}, \emph{input2} e \emph{input3} s\~ao arquivos de entrada contendo descriç\~oes textuais de \emph{workflows} na linguagem definida neste projeto. Não h\'a limite para a quantidade de arquivos de entrada.

            O programa então processa os \emph{workflows} e, caso não haja erros, imprime para a tela:
            \lstinputlisting{sucess}

            E cria os seguintes arquivos de saída:
            \begin{itemize}
                \item workflow1.dot
                \item workflow1.pdf
                \item workflow1.pepa
                \item workflow1\_solution
            \end{itemize}

            No caso de ocorrerem erros no processamento de um \emph{workflow}, a seguinte mensagem ser\'a impressa para a tela:
            \lstinputlisting{fail}

            Onde \emph{workflow1\_traceback} \'e um arquivo contendo a mensagem de erro originalmente produzida, que não foi impressa para a tela para não a poluir. Também podem ser criados alguns arquivos de saída, dependendo do local do c\'odigo onde o erro ocorreu.

            Deve-se notar que o processamento de cada \emph{workflow} \'e feito de forma independente, ou seja, essas mensagens serão impressas para cada \emph{workflow} processado e a ocorr\^encia de erro em um processamente não interfere em nada no pr\'oximo.

  \newpage
  \section{Resultados}

  	O programa, então, executa todos os passos desde a leitura da descrição textual do \emph{workflow} à criação do modelo analítico em PEPA. Para demonstrar as saídas do programa, serão utilizados exemplos de um mesmo experimento.

  	\subsection{Descrição Textual do \emph{Workflow}}
  		Baseada em linguagem DOT, foi definida uma linguagem simples para a descrição textual de \emph{workflows}. Como a intenção dessa linguagem é permitir apenas a descrição de grafos que representem experimentos científicos, e não de qualquer tipo de grafo, ela é muito mais concisa que a linguagem DOT. Os detalhes desta linguagem serão discutidos mais adiante.

	  	\lstinputlisting[caption={Gramática da linguagem de descrição textual de \emph{workflows}},label=lst:gramatica]{gramatica_workflow_entrada.txt}
	  	\lstinputlisting[caption={Exemplo de descrição textual de um \emph{workflow} na linguagem definida},label=lst:linguagem]{entrada.workflow}

    \newpage
	\subsection{Estrutura de Dados Baseada em Grafo}
        
		Utilizando classes em \emph{python}, foi possível implementar diretamente os n\'os (\emph{nodes}), arestas (\emph{edges}) e at\'e mesmo o pr\'oprio \emph{workflow}, bem como diversos \emph{m\'etodos} pertencentes a essas classes que facilitam sua manipulação. Embora suas descriç\~oes sejam simples, não serão exibidos seus c\'odigos, para manter a clareza do texto, e sim representaç\~oes textuais que demonstram os aspectos relevantes da implementação.

        \subsubsection*{N\'o}
            \lstinputlisting[caption={Representação textual de um objeto \emph{Node()}},%
                             label=lst:node]{node}
        
        \subsubsection*{Aresta}
		  \lstinputlisting[caption={Representação textual de um objeto \emph{Edge()}},%
                           label=lst:edge]{edge}

        \subsubsection*{\emph{Workflow}}
            \lstinputlisting[caption={Representação textual de um objeto \emph{Workflow}},%
                             label=lst:workflow]{grafo}

	\newpage
	\subsection{Visualização do \emph{Workflow}}

		A partir do grafo em memória do \emph{workflow}, o programa gera arquivos contendo o código equivalente na linguagem DOT e um \emph{pdf} com a visualização do mesmo.\\


		\begin{minipage}[c]{0.4\textwidth}
			\centering
			\lstinputlisting[caption={Exemplo de descrição do \emph{Workflow} em linguagem DOT},label=lst:dot]{workflow1.dot}
		\end{minipage}%   
		\begin{minipage}[c]{0.7\textwidth}
			\centering
			\includegraphics[width=0.65\linewidth]{workflow1.pdf}
			\captionof{figure}{Exemplo de visualização do \emph{workflow} criada a partir do código em DOT}
			\label{fig:pdf}
		\end{minipage}

	\newpage
    \subsection{Descrição em PEPA}
        Tendo o \emph{workflow} já representado num objeto \emph{Workflow}, o programa então percorre todos seus n\'os, gerando sua descrição em PEPA e, finalmente, gravando essa descrição em um arquivo \emph{.pepa}. 
        \lstinputlisting[caption={Descrição em PEPA do \emph{workflow}.},label=lst:pepa]{workflow1.pepa}

        \newpage
  \section{Análises}
  	\subsection{Descrição Textual do \emph{Workflow}}
  		Como dito anteriormente, a gramática de descrição do \emph{workflow} definida é baseada na linguagem DOT, pois esta é amplamente utilizada na representação de grafos em geral. Porém, essa mesma generalidade implica em maior complexidade sintática, por isso a necessidade da definição de uma nova linguagem, capaz de lidar com as formas mais comuns de \emph{workflows} científicos, mas ainda assim minimalista. Portanto, definimos uma linguagem com as seguintes regras (ver o exemplo do Código \ref{lst:linguagem} para maior esclarecimento):

  		\begin{itemize}
  			\item Toda descrição textual de \emph{workflow} deve ser iniciada com a palavra ``digraph'', para manter similaridade com a linguagem DOT.
  			\item A seguir, pode aparecer o nome do \emph{workflow}, que será usado para nomear os arquivos de saída. \emph{Workflows} sem nome serão nomeados sequencialmente. 
            Ex: \emph{workflow1}, \emph{workflow2}, \emph{workflow3}, ...
  			\item A descrição do \emph{workflow} propriamente dita fica entre chaves.
  			\item Espaços e tabulações são ignorados.
  			\item Todas as linhas da descrição terminam com ponto e vírgula (``;'').
  			\item Não há declaração explícita de nós e arestas, esta é feita de forma implícita.
  			\item Nomes de nós e arestas devem ser alfanuméricos e podem conter o símbolo \emph{underline} ``\_'', mas devem iniciar por uma letra min\'uscula.
  			\item Cada linha representa uma ou mais arestas, utilizando o símbolo ``->'' (seta) para separar o nó de partida de seus nós de destino.
  			\item À esquerda da seta são declarados os nós de partida com seus respectivos atributos após seu nome e entre colchetes: um número, caso seja uma atividade, representando sua taxa de execução, ou uma palavra (OR, XOR ou AND), caso seja um operador, representando seu tipo. Notar que não \'e necessário explicitar o tipo do operador caso este esteja fechando um bloco (no exemplo: XOR2 e AND2), o programa atribuir\'a automaticamente seu tipo. Cada linha só pode conter um nó de partida e seu respectivo atributo.
  			\item À direita são declaradas os nós de destino e seus respectivos atributos, separados por vírgula. Os atributos são números representando a probabilidade daquele caminho (aresta) ser tomado pelos dados e deve estar entre colchetes e antes do nome do nó. Aqui há uma criação de nós ainda não declarados, mas sem atributos, que serão adicionados posteriormente quando estes forem declarados como nós de partida.
  			\item Toda linha deve conter a seta, isto é, não se pode declarar somente um nó em uma linha. No entanto, na declaração de arestas, o nó de destino é criado automaticamente. Isto implica no fato de que nós sem arestas partindo deles não podem ter atributos, exigindo em alguns casos a utilização de um nó especial para simbolizar o final do \emph{workflow}.
  		\end{itemize}

  	
  	\newpage	
  	\subsection{Analisadores Léxico e Sintático}
  		Uma vez decidido partir de uma descrição textual do \emph{workflow}, é necessário que o programa seja capaz de ler e interpretar o texto dado. Logo, é necessário o uso de analisadores léxicos e sintáticos, que têm exatamente essa função.

  		O analisador léxico, ou \emph{lexer}, quebra o texto em pequenos fragmentos, ou \emph{tokens}, seguindo regras definidas pelo usuário. A seguir, passa esses \emph{tokens} ao analisador sintático, ou \emph{parser}, que, também a partir de regras, interpreta o papel de cada \emph{token} na sintaxe geral em relação aos anteriores, executando uma ação específica a cada \emph{token} novo. Apesar de não serem muito simples de se implementar, os analisadores permitem uma grande flexibilidade na definição da gramática do texto.

  		Escolheu-se, então, utilizar o Lex, \emph{A Lexical Analyzer Generator}, e o Yacc, \emph{Yet Another Compiler-Compiler} \cite{web:lexyacc}, geradores de analisadores léxicos e sintáticos, respectivamente. Apesar de antigos (1975 e 1970, respectivamente), ainda são amplamente utilizados através de reimplementações e frequentemente juntos. O que é exemplificado pela biblioteca PLY, \emph{Python Lex-Yacc}, uma implementação recente (2001) inteiramente em \emph{Python}, a qual busca entregar toda a funcionalidade do Lex/Yacc somada a uma extensa verificação de erro.

  		Portanto, ao utilizar a biblioteca PLY, é possível definir uma sintaxe abstrata para o \emph{workflow} independente das outras partes do programa e, ao mesmo tempo, construir o grafo representante em tempo real, isto é, ao longo da leitura do texto. 

  	\subsection{Estrutura de Dados Baseada em Grafo}
  		A escolha de se representar o \emph{workflow} por meio de uma estrutura de dados em memória, em forma de grafo, se dá pela generalidade dessa estrutura, que não depende de nenhuma linguagem, facilitando sua manipulação e eventuais traduções para linguagens específicas. 

  		Por o projeto ser em \emph{Python}, inicialmente se imaginou necessária uma biblioteca desta linguagem que trabalhasse com grafos de uma maneira leve e flexível. Foi encontrada, então, a \emph{python-graph}, que parecia preencher esses requisitos e oferecia um grande número de algoritmos úteis ao se lidar com grafos. No entanto, logo ficou claro que faltavam algumas funções simples e frequentemente era necessário modificar o código da biblioteca. Como isto não era uma boa opç\~ao (o programa deve rodar igualmente em qualquer sistema), decidiu-se abandonar o uso desta biblioteca e se implementar um c\'odigo equivalente no pr\'oprio programa.

      Utilizando apenas tr\^es classes (\emph{Node}, \emph{Edge} e \emph{Workflow}) e alguns algoritmos, foi possível manipular os \emph{workflows} de maneira clara, utilizando uma linguagem de alto n\'ivel, e flexível, permitindo ajustar a estrutura de cada classe conforme necess\'ario.

  		A seguir uma breve descrição de cada classe, bem como suas componentes. Para melhor ilustraç\~ao, ver os C\'odigos \ref{lst:node}, \ref{lst:edge} e \ref{lst:workflow}.

        \newpage
        \subsubsection{Classe \emph{Node}}

            Classe que representa um n\'o, composta por:
            \begin{itemize}
                \item Atributos:
                \begin{itemize}
                    \item \emph{name}: \emph{string} contendo o nome do n\'o
                    \item \emph{type}: \emph{string} contendo o tipo do n\'o (``ACT'', ``OR'', ``XOR'' ou ``AND'')
                    \item \emph{rate}: vari\'avel contendo a taxa de execução do n\'o (n\'umero ou \emph{None})
                    \item \emph{predecessors}: lista contendo todos os n\'os antecessores a este no \emph{workflow}
                    \item \emph{sucessors}: lista contendo todos os n\'os sucessores a este no \emph{workflow}
                \end{itemize}
            \end{itemize}
        \subsubsection{Classe \emph{Edge}}

            Classe que representa uma aresta, composta por:
            \begin{itemize}
                \item Atributos:
                \begin{itemize}
                    \item \emph{tail}: objeto do tipo \emph{Node} contendo o n\'o de partida da aresta
                    \item \emph{head}: objeto do tipo \emph{Node} contendo o n\'o de chegada da aresta
                    \item \emph{prob}: vari\'avel contendo a probabilidade daquela aresta ser percorrida (n\'umero)
                \end{itemize}
            \end{itemize}

        \subsubsection{Classe \emph{Workflow}}

            A classe mais complexa das tr\^es. Representa um \emph{workflow} completo e, além de atributos, cont\'em v\'arios \emph{m\'etodos}. \'E composta por:

            \begin{itemize}
                \item Atributos:
                \begin{itemize}
                    \item \emph{name}: \emph{string} contendo o nome do \emph{workflow}
                    \item \emph{nodes}: dicion\'ario contendo todos os objetos do tipo \emph{Node} que pertencem ao \emph{workflow} indexados por seus respectivos nomes (vari\'avel \emph{name})
                    \item \emph{edges}: dicion\'ario contendo todos os objetos do tipo \emph{Edge} que pertencem ao \emph{workflow} indexados por uma tupla na forma \emph{``(nome do n\'o de partida, nome do n\'o de chegada)''} 
                    \item \emph{pepa}: dicion\'ario contendo listas de \emph{strings} que depois ser\~ao unidas para formar uma \'unica \emph{string} (a descrição em \emph{PEPA} do \emph{workflow}) 
                \end{itemize}
                \item M\'etodos:
                \begin{itemize}
                    \item \emph{add\_node}: funç\~ao que adiciona o n\'o especificado ao \emph{workflow}, atualizando o dicion\'ario \emph{nodes}
                    \item \emph{add\_edge}: funç\~ao que adiciona a aresta especificada ao \emph{workflow}, atualizando o dicion\'ario \emph{edges} e as listas \emph{predecessors} e \emph{sucessors} dos n\'os involvidos
                    \item \emph{has\_node}: funç\~ao que retorna se um n\'o especificado pertence ao \emph{workflow} (retorna \emph{True} ou \emph{False})
                    \item \emph{has\_edge}: funç\~ao que retorna se uma aresta especificada pertence ao \emph{workflow} (retorna \emph{True} ou \emph{False})
                    \item \emph{root\_node}: funç\~ao que retorna o n\'o pertencente ao \emph{workflow} que não tem nenhum antecessor. Presume-se que s\'o exista um n\'o com essa caracter\'istica no \emph{workflow} (n\'o ``raiz'').
                \end{itemize}
            \end{itemize}
 
  	\newpage
  	\subsection{Linguagem \emph{DOT}}
  		Antes realizada automaticamente pela biblioteca \emph{python-graph}, a tradução do \emph{workflow} para linguagem \emph{DOT} teve que ser manualmente implementada. A função responsável por isso (\emph{write}) percorre o \emph{workflow} e cria sua descrição em \emph{DOT} simultaneamente.

      Uma vez obtida tal descrição, cria-se sua representação gráfica, através da biblioteca \emph{Graphviz}, e ambas são salvas em arquivos de saída. Essas etapas são executadas pela funç\~ao \emph{dot\_pdf}, como pode ser visto em seu c\'odigo: 

  		\lstinputlisting[language=Python,caption={Funç\~ao \emph{dot\_pdf com coment\'arios traduzidos}}]{dot.py}

  		Neste projeto foi adotado o formato \emph{pdf}, por sua qualidade, versatilidade e facilidade de inclusão em documentos. Também foram adotadas algumas convenç\~oes para facilitar a visualização do \emph{workflow}, como pode ser verificado no C\'odigo \ref{lst:dot} e na Figura \ref{fig:pdf}, que s\~ao:

        \begin{itemize}
            \item Atividades s\~ao representadas por ret\^angulos
            \item Operadores s\~ao representados por losangos com as seguintes cores:

            \begin{itemize}
                \item \emph{Azul} para operadores do tipo \emph{AND}
                \item \emph{Vermelho} para operadores do tipo \emph{OR}
                \item \emph{Verde} para operadores do tipo \emph{XOR}
            \end{itemize}

            \item Nome de cada componente (atividade ou operador) dentro de seu s\'imbolo
            \item Setas para representar relaç\~oes de preced\^encia entre componentes
            \item Um n\'umero (entre 0 e 1) ao lado de cada seta representando a probabilidade daquele ``ramo'' do \emph{workflow} ser executado. Caso seja 1 (o padr\~ao), ele \'e omitido

        \end{itemize}


  		Apesar de algumas restrições, como a ausência de atributos dos nós nas imagens (Figura \ref{fig:pdf}) e n\~ao distinguir operadores que dividem a linha de fluxo dos que fazem sua junç\~ao, ainda é vantajosa a utilização da linguagem \emph{DOT}, pois suas funções são relativamente simples de implementar e dão ao usuário a possibilidade de verificar se a estrutura em memória corresponde a seu \emph{workflow} original.

    \newpage
  	\subsection{Modelagem Analítica}
  		Uma vez que já existe o grafo na memória, só resta sua tradução para um modelo analítico e então efetuar a análise numérica. Foi escolhido, então, o modelo de álgebra de processos estocástica, por apresentar vantagens em relação a outros modelos, dentre as quais as mais importantes são \cite{web:aboutpepa}:

  		\begin{itemize}
  			\item \emph{Composicionalidade:} a habilidade de modelar um sistema como a interação de subsistemas.
  			\item \emph{Formalismo:} dar um significado preciso para todos os termos na linguagem.
  			\item \emph{Abstração:} a habilidade de construir modelos complexos a partir de componentes detalhadas, desconsiderando os detalhes quando apropriado.
  		\end{itemize}

		    Dentre as linguagens disponíveis, foi usada a \emph{PEPA}, uma álgebra de processos estocásticos bem desenvolvida e que conta com várias ferramentas de apoio, como um complemento para o ambiente integrado de desenvolvimento \emph{Eclipse} e, recentemente, uma biblioteca para a linguagem \emph{Python}, a \emph{pyPEPA}. 

        Essa biblioteca \'e inicialmente utilizada para calcular as probabilidades no regime estacion\'ario de cada um dos estados poss\'iveis do \emph{workflow} descrito no modelo em \emph{PEPA}. A partir destas probabilidades, a \emph{pyPEPA} consegue fornecer o rendimento (\emph{throughput}) das atividades do \emph{workflow} e também a taxa de utilizaç\~ao de seus componentes.

        A \emph{PEPA}, no entanto, \'e uma linguagem muito gen\'erica, desenvolvida para ser capaz de definir in\'umeros tipos de processo. Isto a torna complexa, impondo dificuldades na convers\~ao de modelos de \emph{workflow}. Por este motivo, a ferramente desenvolvida neste projeto converte apenas \emph{workflows} que contenham os operadores \emph{AND} ou \emph{XOR}, mas pode ser estendida para suportar o operador \emph{OR}, mesmo este n\~ao tendo implementação direta em \emph{PEPA}, atrav\'es de uma combinaç\~ao dos outros dois.

        Mesmo assim, a estrutura de um modelo em \emph{PEPA} pode ser resumida em (ver C\'odigo \ref{lst:pepa}):

        \begin{enumerate}
            \item Declaração das constantes que definem as taxas de execução das atividades e dos operadores, bem como as probabilidades dos ramos. No formato: 

            ``\emph{constante} = \emph{express\~ao};''

            Onde \emph{express\~ao} pode ser qualquer equaç\~ao matemática simples envolvendo n\'umeros e/ou \emph{constantes} previamente definidas.

            \item Declaração das equaç\~oes que definem os componentes do \emph{workflow} no formato:

            ``\emph{nome da equaç\~ao} = \emph{componentes};''

            Onde os componentes podem ser:
            \begin{itemize}
                \item Outras equaç\~oes
                \item \emph{Aç\~oes}: s\~ao descritas no formato: 

                ``(\emph{nome}, \emph{constante})''

                Deve-se notar que as aç\~oes aqui representam qualquer n\'o do \emph{workflow}, n\~ao havendo distinç\~ao entre as atividades e operadores utilizados durante o programa.
                \item \emph{Operadores}: usados para controlar o fluxo do \emph{workflow}, s\~ao sempre aplicados entre duas aç\~oes ou equaç\~oes. S\~ao:
                \begin{itemize}
                    \item \emph{Sequ\^encia}: representada por um \emph{ponto} (``.''), indica que uma aç\~ao ou equaç\~ao \'e executada logo ap\'os outra.
                    \item \emph{Escolha}: representada por uma soma (``+''), indica que apenas uma das opç\~oes ser\'a executada.
                \end{itemize}
            \end{itemize}

            \item Definiç\~ao do \emph{workflow} completo, por meios dos paralelismos (indicados pelo operador ``||'') e sincronizaç\~oes (indicadas pelo operador ``<aç\~ao1,aç\~ao2,aç\~ao3...>'', que sincroniza duas equaç\~oes nas aç\~oes especificadas). Esta definiç\~ao deve conter pelo menos a equaç\~ao mestre (geralmente denotada por ``P'') e n\~ao deve terminar em \emph{ponto-e-v\'irgula} (``;'').
        \end{enumerate}

  \section{Conclusões}
  	Neste projeto de iniciação científica, foi desenvolvida uma ferramenta de software que converte de forma autom\'atica modelos de \emph{workflow} em modelos estocásticos e, a partir destes \'ultimos, extrai medidas de desempenho dos \emph{workflows}, facilitando o uso de ferramentas de prediç\~ao por usu\'arios não-especialistas.

    Os modelos de \emph{workflow} recebidos como entrada para o programa são definidos por meio de uma notaç\~ao simples que permite descrever os modelos mais comuns de experimentos cient\'ificos.

    Para a geraç\~ao dos modelos estocásticos, usou-se a linguagem \emph{PEPA}, capaz de descrever uma grande variedade de processos, embora tenha uma sintaxe complexa. A análise numérica desses modelos foi feita por meio da biblioteca \emph{pyPEPA}, uma implementação recente de \emph{PEPA} em \emph{Python}.

    A escolha de \emph{Python} como linguagem de implementação do programa se provou bastante favorável, já que as estruturas de dados e bibliotecas existentes nessa linguagem agilizaram o desenvolvimento das funcionalidades primárias do programa.

  	As etapas realizadas durante o projeto podem ser resumidas em:
  	\begin{itemize}
  		\item O estudo da linguagem \emph{Python} e suas bibliotecas utilizadas
  		\item O estudo dos conceitos relacionados ao tema do projeto, como ``\emph{workflows} científicos'', ``linguagens de modelagem de \emph{workflows}'' e ``modelos estocásticos'', como a PEPA
  		\item Criação de uma linguagem textual simples para a descrição de \emph{workflows}
  		\item Criação de um \emph{lexer} e um \emph{parser} para a leitura da descrição de um \emph{workflow}
  		\item Criação de uma estrutura de dados baseada em grafo em memória que represente um \emph{workflow}
  		\item Gerar arquivos contendo o código em linguagem DOT que representa o grafo em memória e sua visualização
        \item Criação de um algoritmo para a conversão de um grafo de \emph{workflow} para um modelo analítico em \emph{PEPA}
        \item Solução do modelo analítico em \emph{PEPA} e extração dos \'indices de desempenho atrav\'es da biblioteca \emph{pyPEPA}
  	\end{itemize}

    O programa pode ser acessado na p\'agina do projeto \cite{web:script}.

    %\newpage
    \subsection{Dificuldades Encontradas no Projeto}

        A complexidade da sintaxe da \emph{PEPA}, entretanto, dificultou a convers\~ao dos modelos de \emph{workflow}, pois exigiu o tratamento de diversas especificidades desta. Al\'em disso, a implementação recente da \emph{pyPEPA} apresentou diversos obst\'aculos durante o projeto, como sua sintaxe muito restritiva (ela ainda não suporta completamente a sintaxe da \emph{PEPA}) e sua documentaç\~ao escassa. 

        Por estes motivos, as etapas relativas a essas ferramentas ocuparam grande parte do cronograma do projeto, n\~ao sendo poss\'ivel, portanto, realizar outras funcionalidades inicialmente sugeridas neste trabalho, mas que ainda s\~ao interessantes de se implementar, listadas a seguir.
    \subsection{Trabalhos Futuros}
      	\begin{itemize}
            \item Gerar modelos anal\'iticos de \emph{workflows} que contenham o operador \emph{OR}
            \item Definição de uma linguagem para descrição dos recursos que podem ser utilizados por um \emph{workflow}
      		\item Incorporação de informações sobre recursos nos modelos analíticos
            \item Permitir a descrição de \emph{workflows} cuja estrutura não seja ``em blocos''
      	\end{itemize}

  \newpage
  \bibliographystyle{bababbr3}
  \bibliography{relatorio_final}

\end{document}