\documentclass[a4paper,11pt]{article}
\usepackage{listingsutf8}
\usepackage{color}
\usepackage{scalefnt}
\usepackage{cancel}
\usepackage{picinpar}
\usepackage{subfig}
\usepackage{graphicx}
\usepackage{amssymb}
\usepackage{amsmath}
\usepackage[brazilian]{babel}
\usepackage[utf8]{inputenc}
\usepackage[T1]{fontenc}
\usepackage{setspace}
\usepackage{caption}
\usepackage[left=2.5cm,right=2.5cm,top=2.0cm,bottom=1.5cm]{geometry}
\usepackage{siunitx}
\usepackage[colorlinks=true,linkcolor=black,linktoc=all, urlcolor=black]{hyperref}
\usepackage{indentfirst}
\usepackage[nottoc]{tocbibind}
\usepackage[fixlanguage]{babelbib}
\selectbiblanguage{brazilian}
\renewcommand{\lstlistingname}{Código}
\lstset{
    inputencoding=utf8,
    extendedchars=true,
    numbers=left,
    numberstyle=\tiny,
    frame=lines,
    captionpos=b,
    literate=
        {á}{{\'a}}1 {é}{{\'e}}1 {í}{{\'i}}1 {ó}{{\'o}}1 {ú}{{\'u}}1 {ù}{{\`u}}1%
        {Á}{{\'A}}1 {É}{{\'E}}1 {Í}{{\'I}}1 {Ó}{{\'O}}1 {Ú}{{\'U}}1%
        {à}{{\`a}}1 {è}{{\'e}}1 {ì}{{\`i}}1 {ò}{{\`o}}1 {ò}{{\`o}}1%
        {À}{{\`A}}1 {È}{{\'E}}1 {Ì}{{\`I}}1 {Ò}{{\`O}}1 {Ò}{{\`O}}1%
        {ä}{{\"a}}1 {ë}{{\"e}}1 {ï}{{\"i}}1 {ö}{{\"o}}1 {ü}{{\"u}}1%
        {Ä}{{\"A}}1 {Ë}{{\"E}}1 {Ï}{{\"I}}1 {Ö}{{\"O}}1 {Ü}{{\"U}}1%
        {â}{{\^a}}1 {ê}{{\^e}}1 {î}{{\^i}}1 {ô}{{\^o}}1 {û}{{\^u}}1%
        {Â}{{\^A}}1 {Ê}{{\^E}}1 {Î}{{\^I}}1 {Ô}{{\^O}}1 {Û}{{\^U}}1%
        {ã}{{\~a}}1 {ẽ}{{\~e}}1 {ĩ}{{\~i}}1 {õ}{{\~o}}1 {ũ}{{\~u}}1%
        {Ã}{{\~A}}1 {Ẽ}{{\~E}}1 {Ĩ}{{\~I}}1 {Õ}{{\~O}}1 {Ũ}{{\~U}}1%
        {œ}{{\oe}}1 {Œ}{{\OE}}1 {æ}{{\ae}}1 {Æ}{{\AE}}1 {ß}{{\ss}}1%
        {ç}{{\c c}}1 {Ç}{{\c C}}1 {ø}{{\o}}1 {å}{{\r a}}1 {Å}{{\r A}}1%
        {€}{{\EUR}}1 {£}{{\pounds}}1,
}

\date{}
\title{
    Uma Ferramenta de Software para a Predição de Desempenho de Workflows Científicos
}

\author{
    Aluno: Lucas Magno \\
    Bolsista PIBIC da CNPq \\
    Instituto de Física (IF) \\
    \\
    Orientadora: Kelly Rosa Braghetto\\
    Departamento de Ciência da Computação (DCC) \\
    Instituto de Matemática e Estatística (IME) \\
    \\
    Universidade de São Paulo \\
    \href{mailto:lucas.magno@usp.br}{lucas.magno@usp.br}
}

\begin{document}
    \maketitle
    \section*{Resumo}
    Este documento descreve as atividades realizadas durante o período de julho de 2013 a junho de 2014 no âmbito do projeto de iniciação científica do aluno Lucas Magno, número USP 7994983, orientado pela Profa. Dra. Kelly Rosa Braghetto e financiado por uma bolsa PIBIC/CNPq.

    O objetivo principal do projeto foi desenvolver uma ferramenta de software para a conversão automática de modelos de \emph{workflows} em modelos estocásticos na álgebra de processos \emph{PEPA - Performance Evaluation Process Algebra} \cite{web:pepa} . A partir desses modelos estocásticos, é possível extrair predições de desempenho de \emph{workflows}.
    \section*{Abstract}

    \newpage
    \section{Introdução}
        Inicialmente desenvolvidos para automatizar processos industriais e empresariais, os \emph{workflows} se popularizaram e passaram a ser usados na modelagem e automatização de experimentos científicos em diversas áreas da ciência. Um \emph{workflow científico} é a descrição completa ou parcial de um experimento científico em termo de suas atividades, controles de fluxo e dependência de dados \cite{phd:gadelha12}.

        Há várias maneiras de se representar um \emph{workflow científico}, entre elas \emph{grafos direcionados}, \emph{UML \emph{(Unified Modeling Language)}}, \emph{redes de Petri} e \emph{álgebras de processo} \cite{phd:oga11}:. Estes mecanismos de representação são usados para criar modelos que especificam a ordem de execução das atividades dos \emph{workflows}. Além disso, as redes de Petri e as álgebras de processo são linguagens formais, permitindo que se verifiquem propriedades qualitativas e quantitativas dos modelos de \emph{workflow}. Neste trabalho, no entando, somente foram utilizados grafos direcionados e álgebras de processo.

    Para simplificar sua implementação, considera-se que \emph{workflows} sejam compostos por \emph{atividades}, que representam atividades reais de um experimento, e estruturas para descrever o fluxo de controle, como \emph{sequência}, \emph{paralelismo}, \emph{escolha} e \emph{sincronização}, definidas por meio dos operadores \emph{AND} (paralelismo/sincronização), \emph{XOR} (escolha exclusiva/junção) e \emph{OR} (escolha múltipla/junção).

        Por ser comum em experimentos científicos a manipulação de de enormes quantidades de dados e a presença de processos muito demorados, é necessária a análise do desempenho dos \emph{workflows} associados, que pode ser feita através de \emph{medição}, \emph{simulação} ou \emph{modelagem analítica} \cite{phd:kelly11}. Foi escolhida, então, a modelagem analítica, um método preditivo e rápido, implementada por meio de uma álgebra de processos estocástica, a \emph{PEPA}, pois seu uso ainda não foi profundamente explorado para a análise de desempenho preditiva de \emph{workflows} científicos.

    \section{Objetivos}

        Uma desvantagem da modelagem analítica usando \emph{PEPA} é a necessidade da descrição do \emph{workflow} em uma linguagem de modelagem estocástica e utilização de programas específicos para a análise, exigindo do usuário um certo nível de conhecimento sobre álgebras de processo. No entanto, \emph{workflows} científicos são utilizados em diversas áreas da ciência que não necessitam de um grande aprofundamento em computação, o que pode inviabilizar a aplicação deste método.

        O objetivo do trabalho foi facilitar a extração de predições de desempenho para \emph{workflows} científicos. Para isso, foi desenvolvida uma ferramenta de software que gera modelos estocásticos em \emph{PEPA} e sua solução numérica a partir de modelos de \emph{workflows}.

        Os modelos de \emph{workflows} usados como entrada para a ferramenta são descritos textualmente na forma de um grafo dirigido - uma representação simples e que pode ser usada com facilidade por usuários não especialistas. Além do modelo em \emph{PEPA} e sua solução, a ferramenta também gera uma representação gráfica do modelo de \emph{workflow}, que permite que o usuário possa verificá-lo mais facilmente.

    \section{Materiais e Métodos}
        Para que possua um modelo correspondente em \emph{PEPA}, um modelo de \emph{workflow} precisa ser bem estruturado e não possuir ambiguidades semânticas. Por essa razão, neste trabalho consideramos apenas modelos de \emph{workflow} que apresentam somente um ponto de entrada e um ponto de saída, têm sua estrutura em forma de ``blocos''  e não apresentam ciclos, ou laços, o que permite uma implementação mais simples.

        Para automatizar o processo de predição de desempenho, foi implementado um programa na linguagem \emph{Python} (versão 2.7), por flexibilidade, facilidade de aprendizado e grande número de bibliotecas auxiliares. Seu código fonte, detalhes de sua execução, dependências e exemplos de \emph{workflow} de entrada e suas representações em \emph{PEPA} geradas podem ser conferidos na página do projeto \cite{web:script}.

        \newpage
        O programa pode ser resumido nas seguintes etapas:

        \begin{enumerate}
            \item Lê como entrada uma descrição textual de um \emph{workflow} em uma gramática simples baseada na linguagem \emph{DOT} \cite{web:dot} através dos analisadores léxico e sintático gerados a partir da biblioteca \emph{PLY}, \emph{Python Lex-Yacc} \cite{web:ply}.
            \item Gera uma estrutura de dados baseada em grafo na memória representando o \emph{workflow} através de classes explicitamente definidas que permitem a manipulação de nós, arestas e \emph{workflows}.
            \item Gera uma uma descrição do \emph{workflow} de entrada em linguagem \emph{DOT} e sua visualização através da biblioteca \emph{Graphviz} \cite{web:graphviz}.
            \item Gera um modelo analítico do \emph{workflow} em \emph{PEPA}
            \item Gera a solução numérica do modelo analítico e extrai seus índices de desempenho através da biblioteca \emph{pyPEPA} \cite{web:pypepa}, uma implementação recente da \emph{PEPA} em \emph{Python}.
        \end{enumerate}

    \section{Resultados}
    \section{Conclusões}


    \bibliographystyle{bababbr3}
    \bibliography{references}

\end{document}
