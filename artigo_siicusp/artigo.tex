\documentclass[a4paper,10pt]{article}
\usepackage{listingsutf8}
\usepackage{color}
\usepackage{scalefnt}
\usepackage{cancel}
\usepackage{picinpar}
\usepackage{graphicx}
\usepackage{amssymb}
\usepackage{amsmath}
\usepackage[brazilian]{babel}
\usepackage[utf8]{inputenc}
\usepackage[T1]{fontenc}
\usepackage{setspace}
\usepackage{caption}
\usepackage[left=2cm,right=2cm,top=3.0cm,bottom=3cm]{geometry}
\usepackage{siunitx}
\usepackage[colorlinks=true,linkcolor=black,linktoc=all, urlcolor=black]{hyperref}
\usepackage{indentfirst}
\usepackage[nottoc]{tocbibind}
\usepackage[fixlanguage]{babelbib}
\usepackage{caption}
\usepackage{subcaption}
\usepackage{mdwlist}
\usepackage{fancyhdr}
\usepackage{helvet}
\renewcommand{\familydefault}{\sfdefault}

\selectbiblanguage{brazilian}
\renewcommand{\lstlistingname}{Código}
\lstset{
    inputencoding=utf8,
    extendedchars=true,
    numbers=left,
    numberstyle=\tiny,
    frame=lines,
    captionpos=b,
    literate=
        {á}{{\'a}}1 {é}{{\'e}}1 {í}{{\'i}}1 {ó}{{\'o}}1 {ú}{{\'u}}1 {ù}{{\`u}}1%
        {Á}{{\'A}}1 {É}{{\'E}}1 {Í}{{\'I}}1 {Ó}{{\'O}}1 {Ú}{{\'U}}1%
        {à}{{\`a}}1 {è}{{\'e}}1 {ì}{{\`i}}1 {ò}{{\`o}}1 {ò}{{\`o}}1%
        {À}{{\`A}}1 {È}{{\'E}}1 {Ì}{{\`I}}1 {Ò}{{\`O}}1 {Ò}{{\`O}}1%
        {ä}{{\"a}}1 {ë}{{\"e}}1 {ï}{{\"i}}1 {ö}{{\"o}}1 {ü}{{\"u}}1%
        {Ä}{{\"A}}1 {Ë}{{\"E}}1 {Ï}{{\"I}}1 {Ö}{{\"O}}1 {Ü}{{\"U}}1%
        {â}{{\^a}}1 {ê}{{\^e}}1 {î}{{\^i}}1 {ô}{{\^o}}1 {û}{{\^u}}1%
        {Â}{{\^A}}1 {Ê}{{\^E}}1 {Î}{{\^I}}1 {Ô}{{\^O}}1 {Û}{{\^U}}1%
        {ã}{{\~a}}1 {ẽ}{{\~e}}1 {ĩ}{{\~i}}1 {õ}{{\~o}}1 {ũ}{{\~u}}1%
        {Ã}{{\~A}}1 {Ẽ}{{\~E}}1 {Ĩ}{{\~I}}1 {Õ}{{\~O}}1 {Ũ}{{\~U}}1%
        {œ}{{\oe}}1 {Œ}{{\OE}}1 {æ}{{\ae}}1 {Æ}{{\AE}}1 {ß}{{\ss}}1%
        {ç}{{\c c}}1 {Ç}{{\c C}}1 {ø}{{\o}}1 {å}{{\r a}}1 {Å}{{\r A}}1%
        {€}{{\EUR}}1 {£}{{\pounds}}1,
}

\date{}
\title{
    Uma Ferramenta de Software para a Predição de Desempenho de Workflows Científicos\footnote{Este trabalho foi financiado por uma bolsa de iniciação científica CNPq/PIBIC (processo: 155544/2013-6).}
}

\author{
\textbf{\textit{Lucas Magno}}$^\dagger$,\textbf{ \textit{Kelly Rosa Braghetto}}$^\ddagger$\\
\\
\textit{Universidade de São Paulo} / $^\dagger$\textit{Instituto de Física,} $^\ddagger$\textit{Instituto de Matemática e Estatística}\\
\\
\href{mailto:lucas.magno@usp.br}{lucas.magno@usp.br} | \href{mailto:kellyrb@ime.usp.br}{kellyrb@ime.usp.br}
}


\fancyhf{} % sets both header and footer to nothing
\renewcommand{\headrulewidth}{0pt}
\cfoot{SIICUSP 2014 -- 22º Simpósio Internacional de Iniciação Científica e Tecnológica da USP }

\pagestyle{fancy}

\begin{document}

\parindent=0mm

    \maketitle

\vspace{-0.5cm}

    \section*{Resumo}

Os experimentos científicos da atualidade frequentemente lidam com grandes quantidades de dados, cujo processamento demanda muitos recursos computacionais. A análise do desempenho dos sistemas computacionais que automatizam esses experimentos -- os \textit{workflows científicos} -- é importante, pois auxilia no provisionamento dos recursos necessários para garantir sua execução eficiente. Nesse contexto, métodos de análise que possibilitem a predição do desempenho de workflows são de particular interesse. Geralmente, esse tipo de método se baseia em modelos definidos por meio de linguagens formais estocásticas. Entretanto, o uso dessas linguagens requer familiaridade com conceitos complexos, que não são de fácil entendimento para usuários não-especialistas. Este trabalho apresenta a ferramenta de software \texttt{wkf2pepa}, que converte de forma autom\'atica modelos de \emph{workflow} em modelos estocásticos na álgebra de processos \emph{PEPA -- Performance Evaluation Process Algebra} e, a partir destes \'ultimos, extrai índices do desempenho tais como a utilização dos recursos e o rendimento do \emph{workflow} e de suas atividades componentes. Os modelos de \emph{workflow} usados como entrada para a \texttt{wkf2pepa} são definidos em notaç\~ao textual simples e intuitiva, que permite descrever as estruturas de fluxos de atividades mais comumente encontradas nos experimentos cient\'ificos. Assim, a \texttt{wkf2pepa} possibilita que usuários obtenham predições sobre o desempenho de \emph{workflows} de forma prática, sem a necessidade de conhecer detalhes sobre modelagem estocástica e análise numérica.\\

\noindent \textbf{Palavras-chaves:} workflows científicos, predição de desempenho, modelos estocásticos

    \section*{Abstract}


Nowadays, scientific experiments often handle large amounts of data, whose processing requires many computational resources. The performance analysis of the \textit{scientific workflows} (i.e., the computer systems that automate these experiments) is important because it helps in provisioning the necessary resources to ensure efficiency in their execution. In this context, analysis methods that allow the prediction of workflows performance are of particular interest. Generally, this type of method relies on models defined through stochastic formal languages. However, the use of these languages requires familiarity with complex concepts, which are not easily understood by non-expert users. This paper presents \texttt{wkf2pepa} -- a software tool  that automatically converts  workflow models in sthocastic models described in PEPA (\textit{Performance Evaluation Process Algebra}). From the latter, the tool extracts performance indexes such as resource utilization and workflow throughput. The workflow models used as input to the \texttt{wkf2pepa} are defined in a simple and intuitive textual notation, that is expressive enough to describe the structure of the kind of activities flows most commonly found in scientific experiments. Thus, \texttt{wkf2pepa} enables users to obtain predictions about the performance of workflows in a practical way, without the need to know details on stochastic modeling and numerical analysis.\\

\noindent \textbf{Keywords:} scientific workflows, performance prediction, stochastic models

    \thispagestyle{fancy}

    \newpage
    \section*{Introdução}
        Inicialmente desenvolvidos para automatizar processos industriais e empresariais, os \emph{workflows} se popularizaram e passaram a ser usados na modelagem e automatização de experimentos científicos em diversas áreas da ciência. Um \emph{workflow científico} é a descrição completa ou parcial de um experimento científico em termo de suas atividades, controles de fluxo e dependência de dados \cite{phd:gadelha12}.

        Por ser comum em \emph{workflows} científicos a manipulação de enormes quantidades de dados e a presença de processos que consomem muitos recursos computacionais, ferramentas que forneçam predições de desempenho desse tipo de sistema fazem-se necessárias. As predições auxiliam na escolha dos recursos computacionais apropriados para a execução e na identificação de possíveis problemas na modelagem dos \emph{workflows}, possibilitando assim execuções mais eficientes.

        Há várias maneiras de se representar um \emph{workflow científico}, como, por exemplo, por meio de \emph{grafos direcionados}, \emph{UML \emph{(Unified Modeling Language)}}, \emph{redes de Petri} e \emph{álgebras de processos} \cite{phd:oga11}. Estes mecanismos de representação são usados para criar modelos que especificam a ordem de execução das atividades pertencentes aos \emph{workflows}. Linguagens formais como as redes de Petri e as álgebras de processos, vão além de uma simples representação, pois permitem também que se verifique propriedades qualitativas e quantitativas dos modelos de \emph{workflow} nelas representados. Em particular, extensões estocásticas dessas linguagens frequentemente são usadas para a criação de modelos preditivos do desempenho de sistemas computacionais.

        Apesar dos inúmeros benefícios que esses formalismos podem trazer à modelagem de \emph{workflows}, o seu uso na prática impõe algumas dificuldades.  Ele exige que o usuário tenha familiaridade com linguagens e modelos estocásticos (de compreensão pouco intuitiva) e com seus programas de simulação ou análise numérica. No entanto, os \emph{workflows} científicos são criados e manipulados por cientistas das mais diversas áreas da ciência e que, geralmente, não são especialistas em computação.

    \section*{Objetivos}

    Este trabalho aborda o problema da predição de desempenho de \emph{workflows} científicos, propondo uma ferramenta de software que automatiza a geração de predições baseadas em modelos estocásticos. O objetivo da ferramenta é esconder do usuário final a complexidade associada à criação e análise desse tipo de modelo.

     A ferramenta -- chamada de \texttt{wkf2pepa} -- gera automaticamente modelos estocásticos a partir de modelos de \emph{workflows} descritos em uma linguagem bastante simples e de compreensão intuitiva, fácil de ser usada por cientistas de qualquer domínio. A ferramenta obtém a solução numérica dos modelos estocásticos e então extrai índices de desempenho relacionados aos modelos de \emph{workflows} fornecidos como entrada.

    \section*{Materiais e Métodos}

        Os modelos de \emph{workflows} usados como entrada para a ferramenta \texttt{wkf2pepa} são descritos textualmente na forma de um grafo dirigido -- uma representação simples e que pode ser usada com facilidade por usuários não especialistas.
        Para a criação dos modelos estocásticos, optou-se pelo uso da linguagem \emph{PEPA -- Performance Evaluation Process Algebra}, uma álgebra de processos estocástica bem estabelecida e que conta com várias ferramentas de apoio.

         Neste trabalho, considera-se que \emph{workflows} sejam compostos por \emph{atividades}, que representam atividades reais de um experimento e estão associadas a taxas de execução, e estruturas para descrever o fluxo de controle, como \emph{sequência}, \emph{paralelismo}, \emph{escolha} e \emph{sincronização}, definidas por meio dos operadores \emph{AND} (paralelismo/sincronização), \emph{XOR} (escolha exclusiva/junção) e \emph{OR} (escolha múltipla/junção).
        Para que possua um modelo correspondente em \emph{PEPA}, um modelo de \emph{workflow} precisa ser bem estruturado e não possuir ambiguidades semânticas. Por essa razão, neste trabalho são considerados apenas modelos de \emph{workflow} que apresentam somente um ponto de entrada e um ponto de saída, têm sua estrutura em forma de ``blocos'' e não apresentam ciclos, ou laços.

        A \texttt{wkf2pepa} foi implementada na linguagem \emph{Python}. Ela usa a bibllioteca \emph{pyPEPA}~\cite{web:pypepa}, uma implementação recente de um solucionador para \emph{PEPA} em \emph{Python}, para calcular as probabilidades no regime estacion\'ario de cada um dos estados poss\'iveis do \emph{workflow} descrito no modelo em \emph{PEPA}. A partir dessas probabilidades, a \emph{pyPEPA} consegue fornecer o rendimento (\emph{throughput}) das atividades do \emph{workflow} e também a taxa de utilizaç\~ao de seus componentes.

        Além do modelo em \emph{PEPA} e sua solução, a \texttt{wkf2pepa} também gera uma representação gráfica do modelo de \emph{workflow}, que permite que o usuário possa verificá-lo mais facilmente.

        O funcionamento completo da ferramenta \texttt{wkf2pepa} pode ser descrito pelos seguintes passos:
        \begin{enumerate*}
            \item Recebe como entrada uma descrição textual (modelo) de um \emph{workflow} que segue uma sintaxe simples, criada com base na linguagem \emph{DOT} \cite{web:dot};
            \item Por meio de analisadores léxico e sintático gerados com a biblioteca \emph{PLY} (\emph{Python Lex-Yacc}) \cite{web:ply}, lê o modelo de workflow e gera uma representação em memória dele. Essa representação, baseada em grafos, é feita através de classes criadas na ferramenta para a manipulação de nós, arestas e \emph{workflows}.
            \item Gera uma descrição do \emph{workflow} de entrada em linguagem \emph{DOT} e sua representação gráfica (visualização) através da biblioteca \emph{Graphviz} \cite{web:graphviz}.
            \item Gera um modelo analítico (estocástico) do \emph{workflow} na linguagem \emph{PEPA}.
            \item Gera a solução numérica do modelo analítico e extrai os índices de desempenho com a  \emph{pyPEPA} .
        \end{enumerate*}


    \section*{Resultados}

    O código fonte da \texttt{wkf2pepa}, suas dependências, informações sobre o seu uso, exemplos de \emph{workflows} de entrada (e seus respectivos modelos em \emph{PEPA}) podem ser vistos na página do projeto \cite{web:script}.

    Ao processar um \emph{workflow}, a ferramenta cria arquivos de saída contendo a descrição do \emph{workflow} em \emph{DOT}, sua representação gráfica, sua descrição em \emph{PEPA} e os índices de desempenho dela extraídos. Um exemplo de descrição de \emph{workflow} que pode ser usada como entrada para a ferramenta e parte dos resultados de saída gerados para ela são mostrados  na Figura~\ref{fig:exemplo}. Outros exemplos de \emph{workflows} com estruturas mais complexas e seus respectivos resultados encontram-se na página do projeto \cite{web:script}.

\begin{figure}[h]
        \centering

        \begin{subfigure}[b]{0.6\textwidth}
                \footnotesize
                \lstinputlisting[tabsize=1]{example}
                \normalsize
                (a) Descrição do workflow de entrada.\\

                    \scriptsize
                \lstinputlisting[tabsize=1]{example.pepa}
                \normalsize
                (b) Modelo em PEPA gerado.\\

        \end{subfigure}%
        ~~%
        \begin{subfigure}[b]{0.4\textwidth}
                \centering
                \includegraphics[scale=0.55]{example-crop.pdf}

                (c) Visualização criada a \\ partir da saída em DOT
                \label{fig:tiger}
        \end{subfigure}
        \caption{Exemplo de uma execução da \texttt{wkf2pepa}.}
        \label{fig:exemplo}
\end{figure}

    \newpage
    \section*{Conclusões}

    Este trabalho apresenta a ferramenta de software \texttt{wkf2pepa}, que converte de forma autom\'atica modelos de \emph{workflow} em modelos estocásticos e, a partir destes \'ultimos, extrai predições do desempenho dos \emph{workflows}. A predição do desempenho de \emph{workflows} é importante porque auxilia a identificação de problemas em sua modelagem e o provisionamento dos recursos necessários para a execução eficiente desses sistemas.

    Os modelos de \emph{workflow} usados como entrada para a \texttt{wkf2pepa} são definidos por meio de uma notaç\~ao textual simples e intuitiva, que permite descrever as estruturas de fluxos de atividades mais comumente encontradas nos experimentos cient\'ificos. Assim, a \texttt{wkf2pepa} possibilita que usuários obtenham predições sobre o desempenho de \emph{workflows} de forma prática, sem a necessidade de conhecer detalhes sobre modelagem estocástica e análise numérica.

    A \texttt{wkf2pepa} gera modelos estocásticos na álgebra de processos \emph{PEPA}. A ferramenta usa a biblioteca \emph{pyPEPA} para obter a solução numérica dos modelos estocásticos e índices de desempenho tais como a taxa de utilização dos componentes do modelo e o rendimento de cada atividade e do \emph{workflow} como um todo.

    Como trabalhos futuros, pretende-se estender a \texttt{wkf2pepa} para que ela lide com modelos de \emph{workflows} que incluam uma descrição dos recursos disponíveis para a execução. Com isso, os modelos estocásticos gerados e os índices de desempenho extraídos a partir deles fornecerão uma boa aproximação do desempenho real esperado para os \emph{workflows}.


    \bibliographystyle{bababbr3}
    \bibliography{references}

\end{document}
